\appendix
\onecolumn

\renewcommand{\thefigure}{\arabic{figure}} % 只要纯数字编号
\setcounter{figure}{0}
\setcounter{table}{0}
\begin{center}
    \LARGE\textbf{Appendix}
\end{center}

\section{Data Availability Statement}
This project is intended solely for non-commercial academic research. Some visual materials (2D images and video clips) were sourced from publicly available television adaptations of \textit{Journey to the West} and used only within the AI-based interaction reconstruction module. All media were functionally transformed for visualization purposes and fall within the scope of fair use as defined by Article 24 of the Copyright Law of the People's Republic of China.
\section{Artifact and Interaction}

% \subsection{Text and Video Analysis.}
% %图片:AI语义提取“角色+神器+动作(交互)”,示例图(原文+译文+影视剧截图),最终结果得到一个“角色-神器-交互”三级映射图--修改成prompt图,流程更清晰些
% Example of Character–Artifact–Interaction Mapping via Text-Video Semantic Alignment %(Voice Interaction Type
% \begin{figure}[h]
%      \centering
%      \includegraphics[width=0.9\textwidth]{figures/textnew.png}
%      % \caption{Example of Character–Artifact–Interaction Mapping via Text-Video Semantic Alignment (Voice Interaction Type)}
%      \label{fig:Text}
% \end{figure}

{\color{red} \subsection{Interactive Classification}}
%“角色-神器-交互”三级映射表
\label{appendix:b1}
We reviewed the original work and TV series of \textit{Journey to the West}, and summarized the interactive classification of 20 artifacts, which demonstrated premodern HCI modes. The boxes with different colors represent Voice Interaction ~ \includegraphics[height=0.8em]{figures/Map Table/label/Voice.png}, 
Spatial Interaction ~ \includegraphics[height=0.8em]{figures/Map Table/label/Spatial.png}, 
Wearable Interaction ~ \includegraphics[height=0.8em]{figures/Map Table/label/other.png}, 
Gesture Interaction ~ \includegraphics[height=0.8em]{figures/Map Table/label/Gesture.png}, Throwing Interaction ~ \includegraphics[height=0.8em]{figures/Map Table/label/Throwing.png}, Handheld Interaction ~ \includegraphics[height=0.8em]{figures/Map Table/label/hand.png}, Visual Recognition Interaction ~ \includegraphics[height=0.8em]{figures/Map Table/label/Visual.png}, and Non-Interaction ~ \includegraphics[height=0.8em]{figures/Map Table/label/Non.png}. 
%%%%%%%%%%%%%%%%%测试图
\begin{table}[H]
% \caption{}
\centering
\smal{
\begin{tabular}{llc}
%\toprule %头部粗线
%\hline
\multirow{2}*{\textbf{Character}}  & \multirow{2}*{\textbf{Artifact}} & \textbf{Interaction Method}\\
~  & ~ & \includegraphics[height=1.43em]{figures/Map Table/label.png} \\
%\midrule 
%1
Lute Heavenly King & \textbf{A1} Anti fire Cover (Bihuozhao)
& \includegraphics[height=1em]{figures/Map Table/Anti-fire.png}\\
%2
Bodhisattva Manjusri
  & \textbf{A2} Demon revealing Mirror (Zhaoyao Jin)  & \includegraphics[height=1em]{figures/Map Table/Zhaoyao.png}\\
%3
Zhu Bajie & \textbf{A3}  Demon-quelling Pole (Jiuchi Dinpa)
 & \includegraphics[height=1em]{figures/Map Table/Jiuchi.png}\\
%4

%5
Dragon King of the Well & \textbf{A4} Face preserving Pearl (Dinyan Zhu) & \includegraphics[height=1em]{figures/Map Table/DinyanZhu.png}\\
%6

Lao Jun & \textbf{A5} Diamond Jade (Jingang Zhuo)
 &\includegraphics[height=1em]{figures/Map Table/Jinggangzhuo.png}\\
~ & \textbf{A6} Wind settling Pill (Dinfeng Dan)
  & \includegraphics[height=1em] {figures/Map Table/dingfengdan.png}\\
Lingji Bodhisattva & \textbf{A7} Flying Dragon Staff (Feilong Baozhang)
 & \includegraphics[height=1em]{figures/Map Table/Feilong.png}\\
%7
%9
Yellow Eyebrow Ancestor & \textbf{A8} Gold cymbals (Jin Nao)  & \includegraphics[height=1em]{figures/Map Table/Goldcymbals.png}\\

%11
~ & \textbf{A9} Human seed bag (Renzhong Dai)
 &\includegraphics[height=1em]{figures/Map Table/RenzhongDai.png}\\

 Sai Tai Sui & \textbf{A10} Golden Bells (Zijin Lin)
 & \includegraphics[height=1em]{figures/Map Table/ZijinLin.png}\\
%13
Princess Iron Fan & \textbf{A11} Plantain leaf Fan (Bajiao Shan)
 & \includegraphics[height=1em]{figures/Map Table/baojiaoshan.png}\\
%14
Golden Horn King \& Silver Horn King & \textbf{A12} Red gourd (Zijin Hong Hulu)
  & \includegraphics[height=1em]{figures/Map Table/Redgourd.png}\\
  %15
~  & \textbf{A13} Rope (Huangjin Sheng)

 & \includegraphics[height=1em]{figures/Map Table/huangjinshen.png}\\
~  & \textbf{A14} Jade Vase (Yangzhiyu Jinpin)
 & \includegraphics[height=1em]{figures/Map Table/yangzhiyu.png}\\

Red Boy &\textbf{A15}  Gold Band (Jin Gu)
  & \includegraphics[height=1em]{figures/Map Table/Goldband.png}\\
%16 孙悟空
Sun Wukong & \textbf{A16} Somersault Cloud (Jindou Yun)

 &\includegraphics[height=1em]{figures/Map Table/Jindouyun.png}\\

~ & \textbf{A17} Gold Banded Cudgel (Ruyi Jingu Bang)
& \includegraphics[height=1em]{figures/Map Table/Ruyibang.png}\\
%17
~ & \textbf{A18} Tight Band (Jingu Er)
 & \includegraphics[height=1em]{figures/Map Table/TightBand.png}\\
%18
King of Wuji Kingdom & \textbf{A19} White Jade Bowl (Baiyu Yu'er)

 & \includegraphics[height=1em]{figures/Map Table/baiyuYuer.png}\\
%19

  %20
   \vspace{0.5em}
The Three Great Kings of Shituoling & \textbf{A20} Yin-yang Vital Principles Jar (Yinyang Erqi Ping)
 & \includegraphics[height=1em]{figures/Map Table/yin-yang.png}\\
%\bottomrule %添加表格底部粗线
~ & ~
 & \includegraphics[height=2.26em]{figures/Map Table/Num.png}\\
\end{tabular}
}
\end{table}
\vspace{-1em}
\subsection{3D Artifacts Models}
%这个地方会放20个3D神器模型的图片 一张完整的图
\vspace{-1em}
\begin{figure*}[h]
    \centering
    \includegraphics[width=0.82\linewidth]{figures/3D models/Models.png}
    \caption{The 3D models of the 20 selected artifacts}
    \label{fig:enter-label}
\end{figure*}

\newpage
\subsection{Supplementary Interface Visuals}

\begin{figure}[htbp]
  \centering
  \begin{subfigure}[t]{0.47\textwidth}
    \centering
    \includegraphics[width=\textwidth]{figures/Web/End.png}
    \caption{Before activation}
    \label{fig:explore_before}
  \end{subfigure}
  \hspace{0.05\textwidth}
  \begin{subfigure}[t]{0.47\textwidth}
    \centering
    \includegraphics[width=\textwidth]{figures/Web/After.png}
    \caption{After activation: Plantain Leaf Fan (Bajiao Shan)}
    \label{fig:explore_after}
  \end{subfigure}
  \caption{Free exploration on the character-artifact interaction network. Users can explore the system freely before and after activation.}
  \label{fig:explore_combined}
\end{figure}


\section{Evaluation Instruments}
\subsection{Questionnaire Overview}
A concise list of post-experience survey questions covering multiple aspects of the system.
\begin{table}[H]
% \caption{We rev}
\centering
\begin{tabular}{lll}
\toprule
\textbf{Section} & \textbf{No.} & \textbf{Question}  \\
\hline
\textbf{Demographics} & \textbf{Q1} & Age Group \\
~ & \textbf{Q2} & Gender \\
~ & \textbf{Q3} & Educational Background\\
% ~ & \textbf{Q4} & Major\\
~ & \textbf{Q4} & Previous knowledge about JttW\\
\vspace{0.5em}
~ & \textbf{Q5} & Previous knowledge about HCI methods\\
%%%%%%%%%%%%
\textbf{Interactive Device Experience} & \textbf{Q6} & What interactive devices have you used? %(Physical, Touchscreen-based, Motion-sensing or body-tracking, Projection, AR, VR.) 
\\
~ & \textbf{Q7} & Which interaction types do you prefer?\\
~ & \textbf{Q8} & What features do you want in future devices?\\
\vspace{0.5em}
~ & \textbf{Q9} & Are you willing to try new devices?\\
%%%%%%%%%%
\textbf{Interest in JttW} & \textbf{Q10} & What are you most interested in? (Interest areas)\\
~ & \textbf{Q11} & How would you like to explore JttW? (Learning formats)\\
\vspace{0.5em}
~ & \textbf{Q12} & Want to explore JttW via interaction? (Willingness)\\
%%%%%%%%%%%%
\textbf{Project Module Feedback} & \textbf{Q13} & Satisfaction with facial recognition and character matching\\
~ & \textbf{Q14} & Satisfaction with interaction mode hypothesis classification\\
~ & \textbf{Q15} & Satisfaction with multimodal display (original text, video, 3D model, donut chart).\\
% ~ & \textbf{Q17} & Embedded information synchronized with the video\\
% ~ & \textbf{Q18} & Satisfaction with free exploration on network\\
% ~ & \textbf{Q19} & Interactive and rotatable 3D artifact models\\
% \vspace{0.5em}
~ & \textbf{Q16} & Satisfaction with network exploration (through the character–artifact semantic map) \\
%%%%%%%%%%%%
\vspace{0.5em}
~	& \textbf{Q17} & The most engaging part of the experience \\
\textbf{Overall Experience}	& \textbf{Q18} & Easy to use \\
~ & \textbf{Q19} & Interestingness \\
~ & \textbf{Q20} & Enhance immersion \\
% ~ & \textbf{Q25} & Enhance independent exploration \\
~ & \textbf{Q21} & Intention to continue using \\
~ & \textbf{Q22} & Recommendation intention \\
~ & \textbf{Q23} & Overall satisfaction \\
~ & \textbf{Q24} & Knowledge enhancement \\
\bottomrule
\end{tabular}
\end{table}

\newpage
\subsection{Interview Outline}
A semi-structured interview guide conducted after the interactive experience, focusing on users’ first impressions, sense of immersion, multimodal engagement, and interaction reasoning processes.
\begin{table}[H]
% \caption{We rev}
\centering
\begin{tabular}{ll}
\toprule
\textbf{Section} & \textbf{Question} \\
\hline
\textbf{First Impressions} & What was your overall first impression of the system? \\
~ & Which feature caught your attention first, and why? \\
\vspace{0.5em}
~& What novelty did the AI face-matching add to your experience? \\
%%%%%
\textbf{Interaction Reasoning} & How did you guess the artifact's interaction method? \\
~ & Did the “guess-then-reveal” flow enhance engagement or recall? \\
~ & Which artifact stood out most, and for what reason? \\
~ & Did you want to explore more artifacts/roles? What motivated you? \\
~ & Did you feel immersed? Which part contributed most? \\
\vspace{0.5em}
~ & Is this format more memorable than reading or watching? \\
%%%%%%%%%%
\textbf{3D/Multimodal Content} & Was the artifact info (text/image/diagram) clear and helpful? \\
~ & How was the 3D interaction (rotate, zoom in, zoom out, lift, click)? \\
\vspace{0.5em}
~ & How did 3D models help compared to 2D or text? \\
%%%%%%%%%%%%%%%
\textbf{Cultural Understanding} & For non-Chinese users: Did it help cultural understanding? \\
~ & Would you recommend this system to others? Why? \\
~ & Would you want similar systems for other cultures or myths? \\
\vspace{0.5em}
~& Summarize your overall impression and imagined future use. \\
%%%%%%%%%%
\textbf{Usability Feedback} & Did you face any usability issues (e.g., unclear buttons)? \\
~ & Was the interface layout clear? Any overload or gaps? \\
~ & What features would you like to add in future versions? \\
~ & How do you evaluate the system’s visual and color design? \\

\bottomrule
\end{tabular}
\end{table}

% Etiam commodo feugiat nisl pulvinar pellentesque. Etiam auctor sodales
% ligula, non varius nibh pulvinar semper. Suspendisse nec lectus non
% ipsum convallis congue hendrerit vitae sapien. Donec at laoreet
% eros. Vivamus non purus placerat, scelerisque diam eu, cursus
% ante. Etiam aliquam tortor auctor efficitur mattis.

% \section{Online Resources}

% Nam id fermentum dui. Suspendisse sagittis tortor a nulla mollis, in
% pulvinar ex pretium. Sed interdum orci quis metus euismod, et sagittis
% enim maximus. Vestibulum gravida massa ut felis suscipit
% congue. Quisque mattis elit a risus ultrices commodo venenatis eget
% dui. Etiam sagittis eleifend elementum.

% Nam interdum magna at lectus dignissim, ac dignissim lorem
% rhoncus. Maecenas eu arcu ac neque placerat aliquam. Nunc pulvinar
% massa et mattis lacinia.

\end{document}
\endinput
%%
%% End of file `sample-sigconf.tex'.
